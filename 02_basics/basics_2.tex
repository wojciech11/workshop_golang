\documentclass[11pt, letterpaper]{article}
\usepackage[utf8]{inputenc}
\usepackage{hyperref}
\usepackage{enumerate}
\usepackage{listings}
\usepackage{fancyvrb}
\usepackage{kotex}
\usepackage{polski}
\usepackage[document]{ragged2e} % keep all left
\usepackage[english]{babel}

\usepackage{minted} % yaml syntax highlighting

\newenvironment{markdown}%
    {\VerbatimEnvironment\begin{VerbatimOut}{tmp.markdown}}%
    {\end{VerbatimOut}%
        \immediate\write18{pandoc tmp.markdown -t latex -o tmp.tex}%
        \input{tmp.tex}}

\providecommand{\tightlist}{%
  \setlength{\itemsep}{0pt}\setlength{\parskip}{0pt}}

\newcommand*{\email}[1]{\href{mailto:#1}{\nolinkurl{#1}} } 

\title{Golang Programming Workshop\\Basics 2\\{ \small \href{https://creativecommons.org/licenses/by/4.0/}{CC BY 4.0} }  }
\author{Wojciech Barczynski}
\date{}
%\\(wbarczynski.pro@gmail.com)


\begin{document}
\selectlanguage{english}

\begin{titlepage}
\maketitle
\end{titlepage}

\tableofcontents
\pagebreak

%%%%%%%%%%%%%%%%%%%%%%%%%%%%
\section{Golang project layout}

How to think about the Golang project layout:

\begin{itemize}%
\item No rigid project structure;
\item The best practice is to keep things simple;
\item As project grows, you can move to more sophisticated directory structure.
\end{itemize}%

Iteration 1:

\begin{verbatim}
|- _your_package_1
|- _your_package_2
|- _your_package_3
|- vendor/
|- main.go
|- go.mod
|- go.sum
\- README.md
\end{verbatim}

Iteration 2:

\begin{verbatim}
|- cmd/
|  \- _your_app
|       \- main.go
|- _your_package_1
|- _your_package_2
|- vendor/
|- go.mod
|- go.sum
\- README.md
\end{verbatim}

Iteration 3:

\begin{verbatim}
|- cmd/
|  \- _your_app
|       \- main.go
|- _your_package_1
|- _your_package_2
|- vendor/ # go mod download
|- tools/  # tools
|- go.mod
|- go.sum
\- README.md
\end{verbatim}

Iteration 4 - \verb|pkg| might be a good idea if you have other technologies in~your repository:

\begin{verbatim}
|- cmd/
|  \- _your_app
|       \- main.go
|- _your_package_1
|- _your_package_2
|- pkg/
|- vendor/ # go mod download
|- tools/  # tools
|- go.mod
|- go.sum
\- README.md
\end{verbatim}

Good source of inspiration:

\begin{itemize}%
\item \href{https://peter.bourgon.org/go-in-production/}{peter.bourgon.org/go-in-production/} (\href{https://www.youtube.com/watch?v=PTE4VJIdHPg}{video});
\item \href{https://github.com/traefik/traefik}{github.com/traefik/traefik}.
\end{itemize}%

%%%%%%%%%%%%%%%%%%%%%%%%%%%%
\section{Naming conventions in Golang}

Naming conventions:

\begin{itemize}%
\item camelCase (\href{https://go.dev/doc/effective\_go\#mixed-caps}{go.dev/doc/effective\_go\#mixed-caps}): variables, function names;
\item Acronyms should be all capitals, as in ServeHTTP and IDProcessor;
\item file names - \verb|snake_case|;
\item package names - : \begin{itemize}%
\item following the business domain,
\item \href{https://go.dev/doc/effective\_go\#names}{go.dev/doc/effective\_go\#names},
\item \href{https://go.dev/blog/package-names}{go.dev/blog/package-names}.
\end{itemize}%
\end{itemize}%

All the best practises for the naming variables in other languages apply to Golang:

\begin{itemize}%
\item Consistent (easy to guess),
\item Short (easy to type),
\item Accurate.
\item The greater the distance between a name's declaration and its uses, 
the longer the name should be.
\end{itemize}%

All the best practises for naming functions in other languages apply to Golang as well:

\begin{itemize}%
\item one function one thing,
\item good name,
\item not too long.
\end{itemize}%

\bigskip
\textbf{Notice:} common mistake is to make Golang look like your favorite programming language.

%%%%%%%%%%%%%%%%%%
\section{Errors}
Let's define our own error and see how they work. We will use the standard libraries for handling errors.

Notice \verb|error| interface is:

\begin{minted}[frame=single]{go}
type error interface {
  Error() string
}
\end{minted}

Let's write a program that use the following function:

\begin{minted}[frame=single]{go}
package main

import "fmt"

func Divide(a, b int) (int, error) {
    if b == 0 {
        return 0, fmt.Errorf("can't divide '%d' by zero", a)
    }
    return a / b, nil
}
\end{minted}

%%%%%%%%%%%%%%%%%%%%%%%%%%%%%%%%%%%%%%%%%%%%%%%%
\subsection{Convention}

Handling errors:

\begin{itemize}%
\item No \verb|try|/\verb|catch| in Golang,
\item Idiomatic way: \begin{minted}[frame=single]{go}
if err != nil {
	// often wrapped
	// not shown below
	return err
}

// or return
if err != nil {
	// logging the error
	return
}
\end{minted}
\item Returned as the last value: \begin{minted}[frame=single]{go}
func Divide(a, b int) (int, error) {
}
\end{minted}
\item error messages are usually written in lower-case and don’t end in punctuation.
\end{itemize}

%%%%%%%%%%%%%%%%%%%%%%%%%%%%%%%%%%%%%%%%%%%%%%%%
\subsection{Sentimental Errors}

An example of such errors are: \mintinline{go}{sql.ErrNoRows} and \mintinline{go}{io.EOF}, they are declared as:

\begin{minted}[frame=single]{go}
package sql

var ErrNoRows = errors.New("sql: no Rows available")
\end{minted}

The advantage? It is simple to handle:

\begin{minted}[frame=single]{go}
err := db.QueryRow("SELECT * FROM users WHERE id = ?", userID)

// new way, in the comment the old way:
if errors.Is(err, sql.ErrNoRows) { //err == sql.ErrNoRows {
  // an error we know
}  else if err != nil { 
  // another error
}
\end{minted}

or

\begin{minted}[frame=single]{go}
err := db.QueryRow("SELECT * FROM users WHERE id = ?", userID)

if err != nil {
	switch {
	// preferable way:
	case errors.Is(err, sql.ErrNoRows):
		// ...
	default:
		// ...
	}
}
\end{minted}

Advantages? Simple. Disadvantages? \begin{itemize}%
\item Not too much info,
\item they become a part of your package {\small API}.
\end{itemize}%

\bigskip

Please change the implementation of the Employee Mgmt application using \emph{Package errors}\footnote{\href{https://godoc.org/github.com/pkg/errors}{https://godoc.org/github.com/pkg/errors}}: \begin{itemize}%
\item return an error when the employee cannot take hollidays.
\end{itemize}%


You will find the application in the github repo:\\
\href{https://github.com/wojciech11/workshop\_golang/tree/master/02\_basics/examples/employee\_mgmt}{02\_basics/examples/employee\_mgmt}.

%%%%%%%%%%%%%%%%%%%%%%%%%%%%%%%%%%%%%%%%%%%%%%%%
\subsection{Custom Error Types}

Use the following example of a custom error type to change the errors implementation in the \verb|Employee| application:

\begin{minted}[frame=single]{go}
type ParsingError struct {
    IncorrectValue string
}

func (e *ParsingError) Error() string {
    return fmt.Sprintf("Cannot parse %v: internal error",
      e.IncorrectValue)
}
\end{minted}

if we use our custom error code, our code will get more complicated:

\begin{minted}[frame=single]{go}
// old way
if err := Foo(); err != nil {
    switch e := err.(type) {
    case *ParsingError:
        // handling the error
    default:
        log.Println(e)
    }
}
\end{minted}

\begin{minted}[frame=single]{go}
// modern way
err := Foo()

var target = &ParsingError{}

if errors.As(err, &target) {
	// handling the error
}
\end{minted}

Advantages? Disadvantages?
\bigskip

Notice that you can customzie both \mintinline{go}{errors.Is} and \mintinline{go}{errors.As}\footnote{\href{https://go.dev/blog/go1.13-errors}{go.dev/blog/go1.13-errors}}.

%%%%%%%%%%%%%%%%%%%%%%%%%%%%%%%%%%%%%%%%%%%%%%%%
\subsection{Opaque errors}

The idea:

\begin{itemize}
\item return your own errors,
\item provide functions to determine what has happended.
\end{itemize}

An example for Dave Cheney blog \footnote{\href{https://dave.cheney.net/2016/04/27/dont-just-check-errors-handle-them-gracefully}{https://dave.cheney.net/2016/04/27/dont-just-check-errors-handle-them-gracefully}}:

\begin{minted}[frame=single]{go}
type temporary interface {
        Temporary() bool
}
 
// IsTemporary returns true if err is temporary.
func IsTemporary(err error) bool {
        te, ok := err.(temporary)
        return ok && te.Temporary()
}
\end{minted}

\pagebreak

%%%%%%%%%%%%%%%%%%%%%%%%%%%%%%%%%%%%%%%%%%
\subsection{Wrapping and Unwrapping errors}

You should use this approach to errors as your default along with the custom type errors. Notice \verb|%w| (stands for wrapping) and \mintinline{go}{errors.Is}  in the code below:

\inputminted[frame=single]{go}{examples/wrap_errors/main.go}

From \href{https://go.dev/blog/go1.13-errors}{go.dev/blog/go1.13-errors}: ,,\emph{when adding additional context to an error, either with fmt.Errorf or by implementing a custom type, you need to decide whether the new error should wrap the original. There is no single answer to this question; it depends on the context in which the new error is created. Wrap an error to expose it to callers. Do not wrap an error when doing so would expose implementation details.''}

\bigskip

Your task -  implement this error handling in your Employee management application.

%%%%%%%%%%%%%%%%%%%%%%%%%%%%%%%%%%%%%%%%%%
\subsection{Stack trackes}

With help of the package errors\footnote{\href{https://godoc.org/github.com/pkg/errors}{https://godoc.org/github.com/pkg/errors}}, we can provide support for the stacktraces and handling error causes. Let's see how to implement, a stacktrace support for our application: 

\begin{minted}[frame=single]{go}
package main

import (
  "fmt"
  "github.com/pkg/errors"
)


type stackTracer interface {
  StackTrace() errors.StackTrace
}

var errProcess = errors.New("boom")

func processData() error {
  return errProcess
}

func main() {
  err := processData()
  wrappedErr := errors.Wrap(err, "processing failed")
  fmt.Printf("%v", wrappedErr)

  if err, ok := wrappedErr.(stackTracer); ok {
    fmt.Printf("%+v", err.StackTrace())
  }
}
\end{minted}

Unwrapping:

\begin{minted}[frame=single]{go}
package main

import (
  "fmt"
  "net"

  "github.com/pkg/errors"
)

type myErrProcess error

var errProcess myErrProcess = errors.New("boom")

func processData() error {
  return errProcess
}

func main() {
  errData := processData()
  wrappedErr := errors.Wrap(errData, "processing failed")

  _, ok := errors.Cause(wrappedErr).(net.Error)
  if ok {
    fmt.Printf("net.Error")
  }

  errP, ok := errors.Cause(wrappedErr).(myErrProcess)
  if ok {
    fmt.Printf(errP.Error())
  }
}
\end{minted}

%%%%%%%%%%%%%%%%%%%%%%%%%%%%%%%%%%%%%%%%%%%%%%%%%%%%%%%%
\section{Defer, Panic, and Recover}

There is a way to recover from the panic, when we use \mintinline{go}{defer}, \mintinline{go}{panic}, and \mintinline{go}{recover}. We are not going to cover it in the introduction course.

If you cannot wait, check \href{https://blog.golang.org/defer-panic-and-recover}{a defer-panic-and-recover blog post on golang.org} and \href{https://github.com/golang/go/wiki/PanicAndRecover}{the Golang wiki panic and recover article}.

%%%%%%%%%%%%%%%%%%%%%%%%%%%%%%%%%%%%%%%%%%%%%%%%%%%%%%%%%%%%%%%%%%%%%%%%%%%
\section{Error best practices}

Few more recommendations:

\begin{itemize}
\item Handle errors once,
\item Error, keep on the left,
\item Notice: standard errors do not come with stacktraces,
\item Read \begin{itemize}
\item An article from 8thlight \footnote{\href{https://8thlight.com/blog/kyle-krull/2018/08/13/exploring-error-handling-patterns-in-go.html}{https://8thlight.com/blog/kyle-krull/2018/08/13/exploring-error-handling-patterns-in-go.html}},
\item Blog post on go erros - \href{https://go.dev/blog/go1.13-errors}{go.dev/blog/go1.13-errors}.
\end{itemize}
\end{itemize}

\bigskip

Libraries:

\begin{itemize}
\item \href{https://github.com/hashicorp/go-multierror}{github.com/hashicorp/go-multierror}
\item \href{https://github.com/go-errors/errors}{github.com/go-errors/errors}
\item \href{https://github.com/pkg/errors}{github.com/pkg/errors}
\end{itemize}

%%%%%%%%%%%%%%%%%%
\pagebreak
\section{Tests}

Create a simple test in a file -- \mintinline{bash}{main_test.go}. To run tests: 
\begin{minted}[frame=single]{bash}
go test .
\end{minted}

Here is the test:

\begin{minted}[frame=single]{go}
package main

import (
  "fmt"
)

func add(a int, b int) int {
  return a + b
}

func main() {
  fmt.Println(add(10,20))
}

func TestAdd(t *testing.T) {
  if add(10,25) != 20 {
    t.Fatal("Boom!")
  }
}
\end{minted}

\pagebreak
\subsection{Table-driven tests}

1. Create a project \verb|workshop-test|:

\verb|main.go|:

\begin{minted}[frame=single]{go}
package main

import (
  "errors"
  "fmt"
)

var ErrUnknownOperation = errors.New("unknown operation")

func Calculate(op string, a int, b int) (int, error) {
  switch op {
  case "+":
    return a + b, nil
  }
  return 0, ErrUnknownOperation
}

func main() {
  r, _ := Calculate("+", 1, 2)
  fmt.Println(r)
}
\end{minted}

\verb|main_test.go|:

\begin{minted}[frame=single]{go}
package main

import (
  "testing"
)

func TestCalculate(t *testing.T) {
  testCases := map[string]struct {
    op       string
    a        int
    b        int
    expected int
  }{
    "simple add": {"+", 1, 3, 4},
  }

  for tname, d := range testCases {
    t.Logf("test: %s", tname)
    r, err := Calculate(d.op, d.a, d.b)
    if err != nil {
      t.Fatalf("%v", err)
    }
    if r != d.expected {
      t.Fatalf("actual %d expected %d", r, d.expected)
    }
  }
}
\end{minted}

Run the tests:

\begin{minted}{bash}
go test .
\end{minted}

Notice: you can add \verb|-race| to turn on the race detector.

Notice: \verb|go clean -testcache| to clean the cache.

\bigskip
2. Add, first the test, support for division.

%%%%%%%%%%%%%%%%%%%%%%%%%%%%%
\subsection{Test with real X}

In the Golang community, we test against a real databases, file systems, etc. With docker is very easy to spin up a database for your tests. Golang developers use mock libraries as a last resort.

%%%%%%%%%%%%%%%%%%%%%%%%%%%%%
\subsection{Tests short and long}

\begin{minted}[frame=single]{go}
if testing.Short() {
  t.Skip("skipping test in short mode.")
}
\end{minted}

%%%%%%%%%%%%%%%%%%%%%%%%%%%%%
\subsection{Integration tests}

The best practice is to use build tags to distinguish integration tests:

\begin{minted}[frame=single]{go}
// +build integration

package service_test

func TestSomething(t *testing.T) {
  if service.IsMeaningful() != 42 {
    t.Errorf("oh no!")
  }
}
\end{minted}

To run:

\begin{minted}{bash}
go test --tags integration ./...
\end{minted}

%%%%%%%%%%%%%%%%%%%%%%%%%%%%%%%%%%%%
\subsection{Recommended libraries}

\begin{itemize}
\item \href{https://github.com/stretchr/testify}{github.com/stretchr/testify} - assertions, better visibility;
\item If you like the BDD style, look into \href{https://github.com/onsi/ginkgo}{ginko} and \href{gomega}{https://github.com/onsi/gomega}.
\end{itemize}

%%%%%%%%%%%%%%%%%%%%%%%%%%%%%%%%%%%%
\subsection{Look ahead}
There is much more:

\begin{itemize}
\item If your functions \href{https://blog.chewxy.com/2018/03/18/golang-interfaces/}{accept interfaces, return structs}, they are easier to test.
\item Check the brilliant blog on \href{https://peter.bourgon.org/go-for-industrial-programming/}{Go for Industrial Programming} and \href{https://www.youtube.com/watch?v=PTE4VJIdHPg}{the corresponding video}.
\end{itemize}

%%%%%%%%%%%%%%%%%%%%%%%%%%%%%%%%%
\pagebreak
\section{Building web app}

Knowing the basics of Golang, let's build a web application.

\subsection{Simplest}

Writing a web server in Golang, thanks to very solid standard library, is faily simple:

\begin{minted}[frame=single]{go}
package main

import (
  "io"
  "log"
  "net/http"
)

func main() {
  hello := func(w http.ResponseWriter, r *http.Request) {
    io.WriteString(w, "Hello World!")
  }

  // Run http server on port 8080
  err := http.ListenAndServe(":8080", http.HandlerFunc(hello))

  // Log and die, in case something go wrong
  log.Fatal(err)
}
\end{minted}

%%%%%%%%%%%%%%%%%%%%%%%%%
\subsection{Multiplexed}

To multiplex, we need to create a Multiplexer:

\begin{minted}[frame=single]{go}
package main

import (
    "io"
    "log"
    "net/http"
)

func main() {
  mux := http.NewServeMux()

  mux.HandleFunc("/hello", func(w http.ResponseWriter,
      r *http.Request) {
    io.WriteString(w, "Hello")
  })
  
  mux.HandleFunc("/world", func(w http.ResponseWriter,
      r *http.Request) {
    io.WriteString(w, "World")
  })

  log.Fatal(http.ListenAndServe(":8080", mux))
}
\end{minted}


%%%%%%%%%%%%%%%%%%%%%%%%%
\subsection{Handler as a struct}

To customize handler, we can create a \verb|struct|

\begin{minted}[frame=single]{go}
type MyHandler struct {
  Greeting string
}

func (h *MyHandler) ServeHTTP(w http.ResponseWriter,
    r *http.Request) {
  fmt.Fprintf(w, "%s, %s!", h.Greeting, r.RemoteAddr)
}

func main() {
  log.Fatal(http.ListenAndServe(":8080", &MyHandler{
    Greeting: "Hello World!",
  }))
}
\end{minted}

\pagebreak

%%%%%%%%%%%%%%%%%%%%%%%%%
\subsection{Sharing data structures among handlers}

The following example shows how to share data among handlers, e.g., database connection details, configs:

\begin{minted}[frame=single]{go}
package main

import (
  "fmt"
  "log"
  "net/http"
)

type App struct {
  ServiceName string
  // Datasource
  // logging config
}

func (app *App) HelloWorld(w http.ResponseWriter,
    r *http.Request) {
  w.WriteHeader(http.StatusOK)
  fmt.Fprintf(w, "Hello World from " + app.ServiceName)
}

func main() {
  app := App{ServiceName: "MyApp"}
  mux := http.NewServeMux()
  mux.HandleFunc("/", app.HelloWorld)

  log.Fatal(http.ListenAndServe(":8080", mux))
}
\end{minted}

\pagebreak

%%%%%%%%%%%%%%%%%%%%%%%%%
\subsection{Reading body}

Extend the previous example to read the data passed with http body:

\begin{minted}[frame=single]{go}
func (h *Handler) ServeHTTP(w http.ResponseWriter,
			r *http.Request) {

	// read body to the buffer
	body, err := io.ReadAll(r.Body)
	if err != nil {  
		panic(err)
	}

	log.Printf("Got from %s: %s: \n", n,
		r.RemoteAddr, body)
}
\end{minted}

Test it:

\begin{minted}{text}
$ curl -d '{"name": "natalia"}' 127.0.0.1:8080 
\end{minted}

%%%%%%%%%%%%%%%%%%%%%%%%%
\subsection{Parse URL}

We have also support for parsing {\small URL} in \href{https://golang.org/pkg/net/url/}{net/url Package}:

\begin{minted}[frame=single]{go}
// "lang=pl"
q := r.URL.Query()
lang := q.Get("lang")
\end{minted}

%%%%%%%%%%%%%%%%%%%%%%%%%
\subsection{Write multilingual hello-world app}

Your task is to build a hello-world web {\small API} app with hard-coded dictionary of hello-world in different languages: \begin{itemize}
\item \verb|pl|: dzień dobry, dobry wieczor
\item \verb|en|: hi, welcome
\item \verb|de|: guten tag
\end{itemize}

Spec:
\begin{itemize}
\item return greetins with user name, when \emph{lang} and \emph{user} come as a {\small GET} param;
\item either {\small GET} parameter \emph{lang} or \emph{user} is missing return return 400;
\item if we do not have any entry for a value in \emph{lang} return 404.
\end{itemize}

Use \verb|curl| or browser to test your web {\small API}:\begin{minted}{bash}
curl '127.0.0.1:3000?lang=en&user=natalia'
\end{minted}

%%%%%%%%%%%%%%%%%%%%%%%%%
\subsection{Testing handlers}

Create tests to cover the edge cases, suse the following code for the start:

\begin{minted}[frame=single]{go}
func TestHandlers(t *testing.T) {
  // Your handler to test
  handler := func(w http.ResponseWriter, r *http.Request) {
    http.Error(w, "Uh huh", http.StatusBadRequest)
  }

  // Create a request
  r, err := http.NewRequest("GET",
    "http://test.com?lang=pl&user=wojtek", nil)

  // Handle request and store result in w
  w := httptest.NewRecorder()
  handler(w, r)

  // Check out
  if w.Code != http.StatusOK {
    t.Fatal(w.Code, w.Body.String())
  }
}
\end{minted}

\bigskip
Task -- test one of your handlers from the previous excercise.

%%%%%%%%%%%%%%%%%%%%%%%%%%%%%%%%%%%%%%%%%%%%%%%%%%%%%%
\subsection{Libaries for web api}

If you want to build more complex web server, you should check one of the existing libraries, for example: \begin{itemize}
\item \href{https://github.com/go-chi/chi}{github.com/go-chi/chi}
\item \href{https://github.com/gin-gonic/gin}{github.com/gin-gonic/gin}
\end{itemize}

%%%%%%%%%%%%%%%%%%%%%%%%%%%%%%%%%%%%%%
\subsection{ReadTimeout and WriteTimeout for http.Server}
Remember that all {\small IO} operations should be cancel-able or timeout-able:

\begin{minted}[frame=single]{go}
srv := &http.Server{
  Addr: "8080",
  Handler: h,
  ReadTimeout: 2s,
  WriteTimeout:2s,
  MaxHeaderBytes: 1 << 20,
}

srv.ListenAndServe()
\end{minted}

%%%%%%%%%%%%%%%%%%%%%%%%%%%%%%%%%%
\subsection{Environment variables}

Our application should get the configuration throught environments variables:

\begin{minted}[frame=single]{go}
package main

import (
  "fmt"
  "os"
)

func main() {
  envValue, found := os.LookupEnv("LISTEN_PORT")
  if ! found {
    envValue = "8080"
  }
  fmt.Printf(envValue)
}
\end{minted}
\bigskip
Notice: you can also use a library for it, e.g.,\\ \href{https://github.com/jessevdk/go-flags}{github.com/jessevdk/go-flags}.

%%%%%%%%%%%%%%%%%%%%%%%%%%%%%%%%%%
\section{Working with {\small JSON}}

Let's change the input in body for our service to:

\begin{minted}[frame=single]{json}
{
  "value": "hi",
  "lang": "en"
}
\end{minted}

To learn how to use marshalling and unmarshalling, let's write a simple program that uses \href{https://golang.org/pkg/encoding/json/}{encoding/json package}:

\begin{minted}[frame=single]{go}
package main

import (
  "encoding/json"
  "fmt"
)

type Employee struct {
  FistName    string `json:"name"`
  LastName    string
  Internal    string `json:"-"`
  Mandatory   int    `json:"mandatory"`
  Zero        int    `json:"zero,omitempty"`
  iDoNotSeeIt int    `json:"notSeen"`
}

func main() {
  input := `{
    "name": "natalia",
    "lastName": "Buss"
  }`

  var empl Employee
  err := json.Unmarshal([]byte(input), &empl)
  if err != nil {
    // ...
    return
  }
  fmt.Println(empl.FistName)
  fmt.Println(empl.LastName)

  empl.Mandatory = 0
  empl.Zero = 0

  out, _ := json.Marshal(empl)
  fmt.Println(string(out))
}
\end{minted}

Notice: you can build your custom Marshaller/Unmarsheller. \verb|json| supports all data types.

\bigskip

Find out what \verb|json.RawMessage| is? What is a use case for it?

%%%%%%%%%%%%%%%%%%%%%%%%%%%%%%%%%%
\subsection{{\small JSON} and composition}

You can use composition to reuse common structures:

\begin{minted}[frame=single]{go}
type Meta struct {
  MasterdataId string `json:"mdId"`
}

func GetName(data bytes.Buffer) {
  // private type
  type person struct {
    Name string `json:name`
    Meta
  }

  var p person
  err = json.Unmarshal(data.Bytes(), &p)
}
\end{minted}

%%%%%%%%%%%%%%%%%%%%%%%%%%%%%%%%%%%%%%
\section{Web app with memory storage}

Build the following application using \href{https://github.com/go-chi/chi}{github.com/go-chi/chi}, so we can add, display, and remove hello messages:

\begin{itemize}
\item \verb|/hello_msg|, \verb|POST| - add new hello message, the format: \begin{minted}[frame=single]{json}
  {"lang": "pl", "msg": "dzień dobry"}
  \end{minted}

\item \verb|/say_hello?user=natalia&lang=en|, \verb|GET| - say hello: \begin{minted}[frame=single]{json}
  {"lang": "en", "msg": "hi Natalia"}
  \end{minted}
\item \verb|/hello_msg/{id}|, \verb|GET| - list all messages: \begin{minted}[frame=single]{json}
  {
    "result": [
       {"lang": "en", "msg": "hi"}
    ]
  }
  \end{minted}
\item \verb|/hello_msg/{id}|, \verb|DELETE| - remove hello message
\end{itemize}

Start with a simple array, later use a map. To test your application:

\begin{minted}[frame=single]{bash}
curl -X POST -H "Content-Type: application/json" \
     -d '{"lang":"en", "msg": "hi"}' 127.0.0.1:8080/hello_msg

curl -X GET 127.0.0.1:8080/hello_msg
\end{minted}

%%%%%%%%%%%%%%%%%%%%%%%%%%%%%%
\pagebreak
\section{Calling remote {\small API}}

\begin{minted}{go}
package main

import (
  "log"
  "net/http"
  "time"
)

func main() {
  c := &http.Client{
    Timeout: 2 * time.Second,
  }

  log.Println("Fetching...")
  resp, err := c.Get(
    "https://mdn.github.io/learning-area/javascript/oojs/json/superheroes.json")

  if err != nil {
    log.Fatal(err)
  }

  // TBD: read the respo
  // and print it out

  defer resp.Body.Close()
}
\end{minted}

Your task is to parse the output. While looking for the best way to parse it, use your writing-tests skills, so you do not {\small DDOS} mdn.github.io.

\pagebreak
%%%%%%%%%%%%%%%%%%%%%%%%%%%%%%
\subsection{Build Hero API Client}

Refactor the previous application and extract fetching list of heros to a \verb|HeroClient|:

\begin{minted}[frame=single]{go}
type HeroClient struct {
  Client *http.Client
}

func (c *HeroClient) GetThem() (string, error) {
  // your code
}

func main() {
  c := &http.Client{
    Timeout: 2 * time.Second,
  }

  hc := HeroClient{Client: c}
  
  // your code to read and display 
  // superheroes JSON
}
\end{minted}

%%%%%%%%%%%%%%%%%%%%%%%%%%%%%%
\subsection{Testing Calling remote {\small API}}

You can also test whether your calls have proper format by using \mintinline{go}{httptest}:

\begin{minted}[frame=single]{go}
package main

import (
  "fmt"
  "net/http"
  "net/http/httptest"
  "testing"

  "github.com/stretchr/testify/assert"
)

func TestHeroClientAPI(t *testing.T) {
  // Send response to be tested
  server := httptest.NewServer(
    http.HandlerFunc(
      func(rw http.ResponseWriter, req *http.Request) {
        // we can check whether the call is right:
        // assert.Equal(t, req.URL.String(), "/some/path") 
        rw.Write([]byte(`OK`))
      }),
  )

  // Close the server when test finishes
  defer server.Close()
  // Use Client & URL from our local test server
  api := HeroClient{Client: server.Client(),
                    Url: server.URL}
  r, err := api.GetThem()
  assert.NilError(t, err)
  fmt.Println(r)
}
\end{minted}

Please refactor your code from previous exercise, add {\small GET} argument, and~write the test.

\bigskip

Notice: you need to install \href{https://github.com/stretchr/testify\#installation}{github.com/stretchr/testify}.

%%%%%%%%%%%%%%%%%%%%%%%%%%%%%%%%
\section{Working with files}

Based on \href{https://gobyexample.com/writing-files}{gobyexample.com/writing-files} and \href{https://gobyexample.com/reading-files}{gobyexample.com/reading-files}:

\begin{markdown}
1. read /etc/passwd and find a line numer with your user
2. transform passwd to json (name, pid, gid, and path) and write to \verb|${HOME}/passwd.json|
\end{markdown}

\pagebreak
%%%%%%%%%%%%%%%%%%%%%%%%%%%%%%%%%
% https://gobyexample.com/command-line-flags
\section{Parsing {\small CLI} args}

For this exercise, we will use an example from \href{https://gobyexample.com/command-line-flags}{gobyexample.com/command-line-flags}:

\begin{minted}[frame=single]{go}
package main

import "flag"
import "fmt"

func main() {

    wordPtr := flag.String("word", "foo", "a string")

    numbPtr := flag.Int("numb", 42, "an int")
    boolPtr := flag.Bool("fork", false, "a bool")

    var svar string
    flag.StringVar(&svar, "svar", "bar", "a string var")

    flag.Parse()

    fmt.Println("word:", *wordPtr)
    fmt.Println("numb:", *numbPtr)
    fmt.Println("fork:", *boolPtr)
    fmt.Println("svar:", svar)
    fmt.Println("tail:", flag.Args())
}
\end{minted}

Task -- your program should prints all files or directories in a given \emph{path}. You should let a user to specify regex for the file or directories names.

\bigskip
How would you test the {\small CLI} app? 

\bigskip

The most popular Golang library for {\small CLI} application is \href{https://github.com/spf13/cobra}{github.com/spf13/cobra}.

%%%%%%%%%%%%%%%%%%%%%%%%%
\section{References}

\begin{markdown}
- https://github.com/golang/go/wiki/CodeReviewComments
- https://golang.com/doc/effective_go.html
- http://devs.cloudimmunity.com/gotchas-and-common-mistakes-in-go-golang
\end{markdown}

\end{document}