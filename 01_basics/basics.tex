\documentclass[11pt, letterpaper]{article}
\usepackage[utf8]{inputenc}
\usepackage{hyperref}
\usepackage{enumerate}
\usepackage{listings}
\usepackage{fancyvrb}
\usepackage{kotex}
\usepackage{polski}
\usepackage[document]{ragged2e} % keep all left
\usepackage[english]{babel}

\usepackage{minted} % yaml syntax highlighting

\newenvironment{markdown}%
    {\VerbatimEnvironment\begin{VerbatimOut}{tmp.markdown}}%
    {\end{VerbatimOut}%
        \immediate\write18{pandoc tmp.markdown -t latex -o tmp.tex}%
        \input{tmp.tex}}

\providecommand{\tightlist}{%
  \setlength{\itemsep}{0pt}\setlength{\parskip}{0pt}}

\newcommand*{\email}[1]{\href{mailto:#1}{\nolinkurl{#1}} }

\title{Golang Programming Workshop\\Basics\\{ \small \href{https://creativecommons.org/licenses/by/4.0/}{CC BY 4.0} }  }
\author{Wojciech Barczynski}
\date{}

\begin{document}
\selectlanguage{english}

\begin{titlepage}
\maketitle
\end{titlepage}

\tableofcontents
\pagebreak

%%%%%%%%%%%%%%%%%%
\section{Agreement}

\bigskip
\textbf{Notice:} No copy\&paste.
\bigskip

%%%%%%%%%%%%%%%%%%%%%
\section{Prerequiments}

%%%%%%%%%%%%%%%%%%%%%
\subsection{Audience}

Expected background knowledge and skills for the workshop:

\begin{itemize}%
\item Have 1-year hands-on experience in other programming language;%
\item Know how to work with the Command Line Interface in Linux or OS.
\end{itemize}%

%%%%%%%%%%%%%%%%%%%%%
\subsection{Your workstation}

In the github repository, you will find the instruction on how to set up your workstation.

\begin{itemize}%
\item Linux or {\small OSX} recommended;%
\item Basic: \begin{itemize}%
    \item Golang;
    \item {\small IDE} or code editor, vscode is a good start;
    \item Git;
    \item Docker.
    \end{itemize}%
\item {\small SQL} and no{\small SQL} exercise (recommended with docker):
\begin{itemize}%
    \item Postgres,
    \item MongoDB.
\end{itemize}
\end{itemize}

Check \href{https://github.com/golang/go/wiki/IDEsAndTextEditorPlugins}{Go Wiki} to see how to configure your favorite editor to write golang programs.

%%%%%%%%%%%%%%%%%%%%%
\subsection{Verify the setup}
Let's run a hello world to check whether you can run go applications on your workstation.

\bigskip
\textbf{Notice:} No copy\&paste, please.

\bigskip
Let's create a simple program to verify our setup:

\begin{minted}[frame=single]{bash}
# here we will create
# all projects
mkdir workspace
cd workspace

#
mkdir first_project
cd first_project

go mod init
go mod init first_project

# to see 
# in the explorer
xdg-open .
\end{minted}

Create \mintinline{bash}{main.go} in your code editor in the catalog \mintinline{bash}{first_app}, it should have the following text:

\begin{minted}[frame=single]{go}
package main

import "fmt"

func main() {
	// 1 tabulator
	fmt.Println("Hello! YOUR_NAME")
}
\end{minted}

Now, let's run it:

\begin{minted}[frame=single]{bash}
# 1. enforce formatting 
# (with IDEs it is automatic):
gofmt -w .

# 2. run with go:
go run main.go

# 3. build and run:
go build .

# notice new binary executable:
ls

  main.go  first_app

# 3. Run the built binary:
./first_app
\end{minted}

\section{Golang Package Manager}

Notice: Golang has a rich standard library, you should first see whether a given functionality is not already provided, before looking for an external library.

Installing new packages, e.g., \href{https://github.com/logrusorgru/aurora}{github.com/logrusorgru/aurora}:

\begin{minted}[frame=single]{bash}
# you should be
# in ~/workspace/first_project
pwd

go get -u github.com/logrusorgru/aurora/v4

# open also in vscode
# notice the *warnings*
cat go.mod

# 
cat go.sum

# download all (used) deps to vendor/
go mod vendor

ls

  vendor/

# good to know:
go mod tidy
\end{minted}

Let's use the library in our application:

\begin{minted}[frame=single]{go}
package main

import "fmt"
import "github.com/logrusorgru/aurora/v4"

func main() {
	fmt.Println("Hello, ", aurora.Magenta("Natalia"))
}
\end{minted}

\bigskip
3. Installing golang apps/tools and adding them to {\small PATH}:

\begin{minted}[frame=single]{bash}
# let's install our *first_app*
go install

# where is it installed?
#
# Let's find it out.
\end{minted}

\begin{minted}[frame=single]{bash}
go env GOPATH

# do we see the *first_app* binary?
ls $(go env GOPATH)

# do we see the *first_app* binary?
ls $(go env GOPATH)/bin
\end{minted}

What do you see when you run \mintinline{bash}{go env}?

\begin{minted}[frame=single]{bash}
# to have available your golang tooling
# in your terminal:
export PATH=$PATH:$GOPATH/bin

# run it
# now whereever you are in the terminal
first_app
\end{minted}

\bigskip
Note down:
\begin{itemize}
    \item What is GOPATH: . . .
    \item What is GOBIN: . . .
\end{itemize}

%%%
\section{Golang Playground}

Open the browser and run our program on golang playground: \href{https://play.golang.org/}{https://play.golang.org/}.

Notice: you can generate a link to your code sample.

%%%%%%%%%%%%%%%%%%%%%
\section{Variable definition}
\bigskip
0. Create new program \verb|hello-world| (directory in \mintinline{bash}{workspace/}). Copy the \verb|main.go| from our \verb|first_project| project. You should start with:

\begin{minted}[frame=single]{go}
package main

import "fmt"

func main() {
  // 1 tabulator
  fmt.Println("Hello! YOUR_NAME")
}
\end{minted}

\bigskip
1. Please extract your name as a variable, use the following definitions and mark the incorrect ones:

\begin{minted}{go}
// 1
var myName string = "Natalia"
var myName = "Natalia"

// 2
myName := "Natalia"

// 3
var myName
myName = "Natalia"

// 4
var myName string
myName = "Natalia"

// 5
var myName string
myName := "Natalia"

// see which combo works:
fmt.Println("Hello!", myName)
\end{minted}

Notice: in Golang, we use **camelCase** for variable names.

\bigskip
2. Mark the myName as \verb|const|.

\bigskip
3. Declare a variable for your home country and city, use the following construct:

\begin{minted}[frame=single]{go}
var (
  x = 10
  y = 20
)
\end{minted}

and print it on the screen.

%%%%%%%%%%%%%%%%%%%%%%%%%%%%
\section{Integers and Floats}

No big surprise here, numbers:

\begin{itemize}
\item \verb|int|, \verb|int8, ..., int64|, byte $\rightarrow$ int8, rune $\rightarrow$ int32
\item \verb|unint|, \verb|uint8, ..., uint64|
\item \verb|float64|, \verb|float32, float64|
\item \verb|complex|
\end{itemize}

Golang does not support automatic conversion between types. Let's experience it.

\bigskip
1. Declare a variable \mintinline{go}{devExpDays} and \mintinline{go}{msg}:

\begin{minted}[frame=single]{go}
package main

import (
  "fmt"
  "reflect"
  "strconv"
)

func main() {
  name := "Natalia"
  devExpDays := 365
  msg := name + " has " + devExpDays + " exp as developer"
  fmt.Println(msg)
}
\end{minted}

Run it. What error message did you see?

\bigskip
2. To make the code running, we should use \mintinline{go}{strconv.Itoa}. Add the following import and call the function:

\begin{minted}[frame=single]{go}
import (
  "strconv"
  "fmt"
)
\end{minted}

\bigskip
3. Now let's go back from string to integer:

\begin{minted}{go}
// imagine, we got it from the user:
devExpYears := "2"

// does it work?
devExpDays := 365 * devExpYears
\end{minted}

to convert \mintinline{go}{devExpYears} use the following code:

\begin{minted}[frame=single]{go}
// the famous error-return
days, err := strconv.Atoi("12020")
if err != nil {
  fmt.Printf("Cannot convert %v", err)
  return
}
\end{minted}

You have more functions to convert from basic types to string and back, check \href{https://golang.org/pkg/strconv}{Package strconv documentation}.

Notice: \mintinline{go}{import (_ "strconv")}.

%%%%%%%%%%%%%%%%%%%%%%%%%%%%%%%%%%%%
\section{Boolean}

Just to note: \mintinline{go}{true} and \mintinline{go}{false}, standard logical operators: \mintinline{go}{&&}, \mintinline{go}{!}, and \mintinline{go}{||} works.

%%%%%%%%%%%%%%%%%%%%%%%%%%%%%%%%%%%%
\section{Math}
Nothing dramatic here. For more advance mathematical functions, you should check the \href{https://golang.org/pkg/math}{golang.org/pkg/math}:
\begin{minted}{go}
import (
  "math"
)
\end{minted}

%%%%%%%%%%%%%%%%%%%%%%%%%%%%%%%%%%%%
\section{Slices and hidden arrays}
In Golang, we use \mintinline{go}{slices}, seldom we use arrays.

\bigskip
1. If you want to defined an array, you specify the length explicitly:

\begin{minted}[frame=single]{go}
arr1 := [...]string{"pa", "rr", "ot"}
arr2 := [3]string{"pa", "rr", "ot"}

fmt.Print(arr1)
fmt.Printf("%v", arr1)
\end{minted}

Slice, an interface of the array, on the other hand we create with:

\begin{minted}[frame=single]{go}
arr1 := [...]string{"pa", "rr", "ot"}

slice1 := []string{"pa", "na", "ma"}
slice2 := arr1[:]
\end{minted}

Notice: \mintinline{go}{relect.TypeOf(arr1)} vs \mintinline{go}{relect.TypeOf(slice1)}. Good to know when reading compilation errors or runtime panics.

\bigskip
2. What is the output?

\begin{minted}[frame=single]{go}
  var three [3]int
  two := [2]int{10, 20}

  three = two

  fmt.Println(three)
  fmt.Println(two)
\end{minted}

\bigskip
3. Let's define our hello world messages and add one more:

\begin{minted}[frame=single]{go}
helloWorld := []string{"dzień dobry", "Hallo", "guten Tag"}
fmt.Printf("A: len: %d cap: %d \n", len(helloWorld),
  cap(helloWorld))

czechia := "Ahoj"
helloWorld = append(helloWorld, czechia)

fmt.Printf("B: len: %d cap: %d \n", len(helloWorld),
  cap(helloWorld))

fmt.Printf("%v\n", helloWorld)
\end{minted}

\bigskip
Note down:
\begin{itemize}
    \item the \verb|len| and \verb|cap| in \verb|A| and \verb|B|: . . .
    \item What has happen when we appended one slice to another?
\end{itemize}

\bigskip
4. Slices of slices. How would you write a one liner to print out:

\begin{itemize}
  \item \mintinline{go}{["dzień dobry"]}: \bigskip
  \item middle hallo messages: \bigskip
  \item all except the last one: \bigskip
  \item just 1st element: \bigskip
  \item just 15th element: \bigskip
\end{itemize}

Hint: use \mintinline{go}{slice[x]}, \mintinline{go}{slice[x:]}, \mintinline{go}{slice[:x]}, and \mintinline{go}{slice[x:y]}.

\bigskip
5. Watch out, the slices might bite your head off. Note, slice has \emph{capacity}, \emph{length}, and \textbf{pointer to the underlaying array} (see \href{reflect.SliceHeader}{https://golang.org/pkg/reflect/\#SliceHeader}):
\begin{minted}[frame=single]{go}
package main

import "fmt"

func main() {
  helloWorld := []string{"dzień dobry", "Ahoj", "Goodmorning"}
  centralEuHello := helloWorld[0:2]
  fmt.Printf("len: %d cap: %d \n", len(centralEuHello),
    cap(centralEuHello))

  centralEuHello[0] = "Dobry Wieczor"
  fmt.Printf("%v\n", helloWorld)
  fmt.Printf("%v\n", centralEuHello)
}
\end{minted}
What is the result?

\bigskip
\bigskip
Replace \mintinline{go}{centralEuHello[0] = "Dobry Wieczor"} by:

\begin{minted}[frame=single]{go}
 centralEuHello = append(centralEuHello, "Dobry Wieczor")
 centralEuHello = append(centralEuHello, "Dobry Wieczor")
\end{minted}

What is the result? What has happend?

\bigskip
\bigskip
\bigskip
It should be not a suprise that:

\begin{minted}[frame=single]{go}
  a := []int{1,3,5,7}
  b := []int{2,4,6}

  a = b

  a[0] = 99

  // prints the same
  fmt.Printf("%+v\n", a)
  fmt.Printf("%+v\n", b)
\end{minted}

\bigskip
6. Let's fix that with the following code snipped:

\begin{minted}[frame=single]{go}
newSlice := make([]string, 2)
copy(newSlice, slice)
\end{minted}

\bigskip
7. We can also use make to create a slice with desired length and capacity:

\begin{minted}[frame=single]{go}
  msgs := make([]string, 2, 20)
  msgs[0] = "Ahoj"
  msgs[1] = "Goodmorning"

  for idx := range msgs {
    fmt.Printf("%s\n", msgs[idx])
  }

  for idx, v := range msgs {
    fmt.Printf("%d, %s\n", idx, v)
  }

  for _, v := range msgs {
    fmt.Printf("%s\n", v)
  }
\end{minted}

\bigskip
8. Note, when a function returns no result, use a \emph{nil slice}:\\ \mintinline{go}{var nilSlice []string}.

%%%%%%%%%%%%%%%%
\section{nil gotchas}

What does \mintinline{go}{fmt.Println(nilSlice==nil)} prints?

\begin{minted}[frame=single]{go}
func main() {
	var nilSlice []string

	fmt.Println(nilSlice==nil)
}
\end{minted}

\bigskip
9. Let's add a nil value for a pointer to integer:

\begin{minted}[frame=single]{go}
//var nilSlice []string

var i *int = nil
fmt.Println(i==nil)
\end{minted}

ok, so both variable are \mintinline{go}{nil}, let's compare them to each other:

\begin{minted}[frame=single]{go}
fmt.Println(nilSlice==i)
\end{minted}

\bigskip
What did happen?

\bigskip
[Homework] read \href{https://golang.org/doc/faq#nil_error}{golang.org {\small FAQ} entry on nill errors} and \href{https://dave.cheney.net/2017/08/09/typed-nils-in-go-2}{Dave Cheney blog post}.

% package main

% import "fmt"
% import "net"

% func g() error {
%   var e net.Error = nil
%   return e
% }

% func k() error {
%   var e *net.AddrError = nil
%   return e
% }

%  func main() {

%   e1 := g()
%   e2 := k()
%   fmt.Print(e1 == e2)
% }

%%%%%%%%%%%%%%%%%%%%%%%%%%%%%%%%%%%%
\section{Control structure: Loops}

The Go for loop is similar to—but not the same as—C's. It unifies for and while and there is no do-while\footnote{\href{https://go.dev/doc/effective\_go\#for}{go.dev/doc/effective\_go\#for}.}. In Golang there is only one loop keyword:

\begin{minted}[frame=single]{go}
  for i := 0; i < 10; i++ {
    fmt.Println(i)
  }

  for i < 10 {
    fmt.Println(i)
    i++
  }
  // also map
  for index, value := range someSlice {
    fmt.Println(index, value)
  }
\end{minted}

\begin{minted}[frame=single]{go}
// Like a C for
for init; condition; post { }

// Like a C while
for condition { }

// Like a C for(;;)
for { }
\end{minted}

%%%%%%%%%%%%%%%%%%%%%%%%%%%%%%%%%%%%
\section{String}

String is a read-only \textbf{slice} of bytes. Go source code is always in {\small UTF}-8.

\bigskip
1. Run the following code, note down the results:

\begin{minted}[frame=single]{go}
const adress := "ul. Przeskok 2"

fmt.Printf("len: %d\n", len(adress))

fmt.Printf("1: %s\n", adress[0:3])
fmt.Printf("2: %c\n", adress[2])
fmt.Printf("3: %s\n", adress[2:])
fmt.Printf("4: %s\n", adress[5:])

fmt.Printf("5: %s\n", adress[16:])
fmt.Printf("6: %s\n", adress[:16])
\end{minted}


\bigskip
2. Use the following example to build your own program printing out your~3~favorite emoicons:

\begin{minted}[frame=single]{go}
var milk = "우유"

for index, runeValue := range milk {
	fmt.Printf("%c (%U) starts at byte position %d\n",
		runeValue, runeValue, index)
}
\end{minted}

\bigskip
3. What does now happen?:

\begin{minted}[frame=single]{go}
const milk = "우유"

  for i := 0; i < len(milk); i++ {
    fmt.Printf("byte: %x at the index %d\n",
      milk[i], i)
  }
\end{minted}

Now let's mix things up, write a short program that prints runes for:\\ \mintinline{go}{const mixed = "wöjtk유wx"}.

\bigskip
Notice: Important packages are \verb|Package strings| (\href{https://pkg.go.dev/strings}{pkg.go.dev/strings}) and \verb|Package unicode/utf8| (\href{https://pkg.go.dev/unicode/utf8}{pkg.go.dev/unicode/utf8}).

%%%%%%%%%%%%%%%%%%%%%%%%%%%%%%%%%%%%
\section{Maps}

Build a program that displays a hello-world message for different languages.

1. Define the map:

\begin{minted}[frame=single]{go}
  helloMsgs := map[string]string{}
  helloMsgs["pl"] = "Dzień Dobry"
  helloMsgs["en"] = "Good morning"
\end{minted}

\bigskip
2. Read input from the user:

\begin{minted}[frame=single]{go}
  helloMsgs := map[string]string{}
  helloMsgs["pl"] = "Dzień Dobry"
  helloMsgs["en"] = "Good morning"

  var lang string
  // read a single world in ASCII
  // skip error handling
  fmt.Scan(&lang)
\end{minted}

\bigskip
2. Print the hello message:

\begin{minted}[frame=single]{go}
if val, ok := helloMsgs[lang]; ok {
  // found
} else {
  // not found
}
\end{minted}

\bigskip
3. Notice, Golang has a cool feature:

\begin{minted}[frame=single]{go}
  helloMsgs := map[string]string{
    "pl": "Dzień Dobry",
    "en": "Good morning",
  }
\end{minted}

We want to have different greetings depending on the time of day:

\begin{minted}[frame=single]{go}
  helloMsgs := map[string]map[string]string{
    "pl": {"morning": "Dzień Dobry"},
    "en": {"morning": "Good morning"},
  }
\end{minted}

Now, users got bored with the same greeting:

\begin{minted}[frame=single]{go}

  helloMsgs := map[string]map[string][]string{
    "pl": {"morning": []string{
      "Dzień Dobry",
      "Piękny poranek",
    }},
    "en": {"morning": []string{
      "Good morning",
      "Morning",
    }},
  }
\end{minted}

Let's make it more readable:

\begin{minted}[frame=single]{go}
  type daytimeGreetings map[string][]string
  type g map[string]daytimeGreetings

  helloMsgs := map[string]daytimeGreetings{
    "pl": {"morning": []string{
      "Dzień Dobry",
    }},
    "en": {"morning": []string{
      "Good morning",
    }},
  }
\end{minted}

You can use this declarative style to define even the most complex {\small JSON} structures.

\bigskip
4. Write a program that randomize the messages, use \emph{math/rand} and \emph{time} packages to initialize random seed.

\bigskip
5. [Homework] Use time information to find out which part of the day we have.

%%%%%%%%%%%%%%%%%%%%%%%%%%%%%%%%%%%%
\section{User defined type}

You can define types over the basic and~composite types:

%%%%%%%%%%%%%%%%%%%%%%%%%%%%%%%%%%%%
\subsection{Type Definitions}

\begin{minted}[frame=single]{go}
package main

import "fmt"

type myInt int

func display(i int) {
  fmt.Printf("%d", i)
}

func main() {
  var i myInt = 12
  i = i + 12
  // how to fix it?
  display(i)
}
\end{minted}

%%%%%%%%%%%%%%%%%%%%%%%%%%%%%%%%%%%%
\subsection{Type Alias Declarations}

Since Go 1.9, we can declare custom type aliases by using the following syntax:

\begin{minted}[frame=single]{go}
package main

import "fmt"

type myInt = int

func display(i int) {
  fmt.Printf("%d", i)
}

func main() {
  var i myInt = 12
  i = i + 12
  display(i)
}
\end{minted}

%%%%%%%%%%%%%%%%%%%%%%%%%%%%%%%%%%%%
\section{fmt.Printf}

Check the \href{https://gobyexample.com/string-formatting}{https://gobyexample.com/string-formatting}.

%%%%%%%%%%%%%%%%%%%%%%%%%%%%%%%%%%%%
\section{Functions}

1. Let's move the logic for displaying the hello message to a function. We will return \verb|false| if we do not support a given language:

\begin{minted}[frame=single]{go}
func displayHello(lang string) (bool) {
    helloMsgs := map[string]string{
    "pl": "Dzień Dobry",
    "en": "Good morning",
    }
    // here code
    return false
}
\end{minted}

\bigskip
2. Let's follow the golang way and return an error:

\begin{minted}[frame=single]{go}
func displayHello(lang string) (err error) {
  // put code here
  err = fmt.Errorf("Unsupported language")
  // .. and here
  return err
}

func main() {
  err := displayHello()
  if err != nil {
    fmt.Printf("Not found!!! %v", err)
    return
  }
}
\end{minted}

\bigskip
3. Functions are the first class citizens in Golang and we often use them as arguments. Let's write new program:

\begin{minted}[frame=single]{go}
func printThings(msg []string, decorator func(string) string) {
  for _, l := range msg {
    d := decorator(l)
    fmt.Println(d)
  }
}

func main() {
  things := []string{"mleko", "cars", "programming"}

  likeThat := func(s string) string {
    return "Ania likes " + s
  }
  printThings(alphabet, likeThat)
}
\end{minted}

Now something more complicated:

\begin{minted}[frame=single]{go}
type letterDecorator func(string) string

//func printLetters(msg []string, decorator func(string) string)
func printLetters(msg []string, decorator letterDecorator) {
  for _, l := range msg {
    d := decorator(l)
    fmt.Println(d)
  }
}

func main() {
  alphabet := []string{"a", "b", "c", "d", "d"}

  printLetters(alphabet, func(s string) string {
    return strings.ToUpper(s)
  })
  printLetters(alphabet, func(s string) string {
    return "::" + strings.ToLower(s) + "::"
  })
}
\end{minted}

\bigskip
4. Notice: we have support for variadic parameters in functions:\\ \mintinline{go}{func printSymbols(msg ...string)}:

\begin{minted}[frame=single]{go}
printSymbols()
printSymbols("a", "z")
printSymbols(alphabet...)
\end{minted}

\bigskip
5. We can move the execution of a function to the end of the scope with \verb|defer|:

\begin{minted}[frame=single]{go}
package main

import "fmt"

func main() {
  defer fmt.Println("booom!")

  for i := 0; i < 10; i++ {
     fmt.Println("tick...")
  }
}
\end{minted}

We will come back to \mintinline{go}{defer} later in this workshop.

%%%%%%%%%%%%%%%
\section{Control structure: if and switch}

\subsection{switch}

The \emph{switch} can work on any data type, you do not need to switch on a value, see: \href{Golang Wiki}{https://github.com/golang/go/wiki/Switch}.

The very common case for \emph{switch} is to check types:

\begin{minted}[frame=single]{go}
func do(v interface{}) string {
  switch u := v.(type) {
  case int:
    return strconv.Itoa(u*2) // u has type int
  case string:
    mid := len(u) / 2 // split - u has type string
    return u[mid:] + u[:mid] // join
  }
  return "unknown"
}
\end{minted}

\subsection{if}

The \emph{If} and \emph{else} works as in other programming languages, except that you can put a language expression:

\begin{minted}[frame=single]{go}
if err := dec.Decode(&val); err != nil {
  // handling error
}
// happy path
\end{minted}

Example from the \href{https://golang.org/src/net/net.go}{Package net}:

\begin{minted}[frame=single]{go}
if nerr, ok := err.(net.Error); ok && nerr.Temporary() {
  // here nerr is an instance of net.Error
  // and the error is Temporary
}
\end{minted}

Let's build an app for writing and reading a file:

\begin{minted}[frame=single]{go}
package main

import (
  "io/ioutil"
  "fmt"
)

func main() {
  // change to /dat1
  fPath := "/tmp/dat1"
  inD := []byte("hello\nWorld\n")
  if err := ioutil.WriteFile(fPath, inD, 0644); err != nil {
    panic(err) // failfast
  }
  fmt.Println("Write was successful!")

  outData, err := ioutil.ReadFile(fPath)
  if err != nil {
     panic(err) // failfast
  }
  fmt.Print(string(outData))
}
\end{minted}

If it works, change \mintinline{go}{ioutil.ReadFile(fPath)} to \mintinline{go}{ioutil.ReadFile(fPath + "x")}. What does happen?


%%%%%%%%%%%%%%%
\section{Pointers}

Remember:

\begin{itemize}
\item Because of the Golang design, your code will work usually faster if you pass small data types by value;
\item Do not be overzealous with the pointers;
\item Maps, slices, and pointers are reference types.
\end{itemize}

Write the following program:

\begin{minted}[frame=single]{go}
package main

import "fmt"

func tryAnswerEverything(i int) {
  i = 45
}

func answerEverything(i *int) {
  *i = 42
}

func main() {
  i := 33

  fmt.Println(i)
  tryAnswerEverything(i)
  fmt.Println(i)

  answerEverything(&i)
  fmt.Println(i)
}
\end{minted}

\textbf{Notice}, we can return in functions pointers to local variables: \mintinline{go}{func Answer() *int}.

%%%%%%%%%%%%%%%%%%%%%%%%%%%%%%
\section{Structures}

In Golang, we do not have \emph{classes}, instead \emph{struct} with \emph{methods}. The language does not support inheritance, favors composition instead.

%%%%%%%%%%%%%%%%%%%%%%%%%%%%%%
\subsection{Structures and Methods}

1. Write a program to manage employees:

\begin{minted}[frame=single]{go}
package main

type Employee struct {
    FirstName   string
    LastName    string
    leavesTotal int
    LeavesTaken int
}

func (empl *Employee) TakeHolidays(days int) error {
  // write an implementation
}

func (empl *Employee) limitExceeded(days int) bool {
  // write an implementation
}

func main() {
  empl := new(Employee)
  empl.FirstName = "Laste"
  empl.LastName = "BB"
  empl.leavesTotal = 26
  fmt.Println(empl.FirstName)
  // simulate here taking day offs
  // and printing the new value.

  // try to call limitExceeded on empl
}
\end{minted}

\bigskip

% Question: How to init the Employee outside our package?

% \begin{minted}[frame=single]{go}
% // hide the implementation
% type employee struct {
% }

% // provide a factory method New or NewEmployee
% func NewEmployee(firstName string, lastName string,
%     leaveDays int) *employee {

%   return &employee{FirstName: firstName, LastName: lastName,
%     totalLeaves : leaveDays}
% }


% \end{minted}

\bigskip

2. Refactor our application and move implementation of the \verb|employee| to a separate package. To do it, create a directory \verb|employee| and create inside \verb|employee.go|:

\begin{minted}[frame=single]{go}
package employee

type Employee struct {
  FirstName   string
  LastName    string
  leavesTotal int
}

// ... the functions
// ...

\end{minted}

Create an instance of \verb|Employee| in \verb|main.go|.

\bigskip

3. What if, we do not want to expose the implementation details to the code in the main. Change the name of the type from \verb|Employee| to \verb|employee|:

\begin{minted}[frame=single]{go}
package employee

type employee struct {
  FirstName   string
  LastName    string
  leavesTotal int
}

// ... rest of the code
\end{minted}

Does your app works? How we should solve this problem.

\bigskip

3. Implement the solution.

% func New(firstName string, lastName string,
%     leaveDays int) *employee {

%   return &employee{FirstName: firstName, LastName: lastName,
%     leavesTotal: 30}
% }

%%%%%%%%%%%%%%%%%%%%%%%%%%%%%%
\section{Pointer receiver vs value receiver}

What is the difference?

\begin{minted}[frame=single]{go}
func (empl *Employee) TakeHollidays(taken int) {
    empl.leavesTaken = empl.leavesTaken + taken
}
\end{minted}

and:

\begin{minted}[frame=single]{go}
func (empl Employee) TakeHollidays(taken int) {
    empl.leavesTaken = empl.leavesTaken + taken
}
\end{minted}

Write a new app to find out.

%%%%%%%%%%%%%%%%%%%%%%%%%%%%%%
\section{Structures and Interfaces}

In Golang, we have duck typing interfaces, it means that the structure has to implement the functions from the interface.

An example, we will build an application that supports two storage types - postgres and mongo:

\begin{minted}[frame=single]{go}
package main

import "db-example-app/postgres"

type DataStore interface {
  GetEmployee(id int) string
}

type App struct {
  ds DataStore
}

func main() {
  app1 := App{ds: &postgres.PsqlStore{}}
  fmt.Print(app1.ds.GetEmployee(12))
}
\end{minted}

We might have two configurable stores, one psql:

\begin{minted}[frame=single]{go}
package postgres

type PsqlStore struct {
}

func (ps *PsqlStore) GetEmployee(id int) string {
  return "psql"
}
\end{minted}

and one, mongodb:

\begin{minted}[frame=single]{go}
package mongo

type MongoStore struct {
}

func (ps *MongoStore) GetEmployee(id int) string {
  return "mongo"
}
\end{minted}

Run the app and verify that the mongoStore works as well. When it works, break it, for e.g., change the type of args in \mintinline{go}{MongoStore}.

%%%%%%%%%%%%%%%%%%%%%%%%%%%%%%
\section{Composition instead of Inheritance}

\begin{minted}[frame=single]{go}
type person struct {
  firstName string
}

func (p person) name() string {
  return p.firstName
}

type employee struct {
  EmployeeID string
  person
}

func main() {
  empl := employee{}
  empl.firstName = "Natalia"
  empl.person.firstName = "Christof"
  fmt.Println(empl.firstName)
}
\end{minted}

Notice: For polymorphism, you need to use \emph{interface}.

%%%
\section{Interfaces vs Functions}

If you get high on interface, do not forget that you can use functions instead. See an example from~\href{https://golang.org/src/net/http/server.go}{net/http}:

\begin{minted}[frame=single]{go}
type HandlerFunc func(ResponseWriter, *Request)

// ServeHTTP calls f(w, r).
func (f HandlerFunc) ServeHTTP(w ResponseWriter, r *Request) {
    f(w, r)
}
\end{minted}

Let's see it on an example:

\begin{minted}[frame=single]{go}
package main

// employee package
type PolicyHandler func(employee *Employee) bool

func (f PolicyHandler) canTakeHolidays(empl *Employee) bool {
  return f(empl)
}

func (f PolicyHandler) HolidayFreeze() bool {
  return true
}

type Employee struct {
  totalLeaves int
  leavesTaken int
  PolicyHandler PolicyHandler
}

func (empl *Employee) TakeHolidays() bool  {
  if empl.PolicyHandler.canTakeHolidays(empl) != true {
    return false
  }
  return true
}
\end{minted}

\begin{minted}[frame=single]{go}
import "fmt"

func main() {
  var newPolicy PolicyHandler = func(empl *Employee) bool {
      return empl.totalLeaves > empl.leavesTaken
  }
  freeze := newPolicy.HolidayFreeze()
  fmt.Println(freeze)

  employee := Employee{12, 20, newPolicy}
  employee.TakeHolidays()
}
\end{minted}

%%%%%%%%%%%%%%%%%
\section{First Web App}

Now, we will build our first web app before going to the 2nd part of the basics. We will use \href{https://github.com/go-chi/chi}{github.com/go-chi/chi}.

1. Create an app: \emph{first\_web\_app}, install the \verb|chi| package.

\bigskip

2. Start with the example from the github:

\begin{minted}[frame=single]{go}
package main

import (
  "net/http"

  "github.com/go-chi/chi/v5"
  "github.com/go-chi/chi/v5/middleware"
)

func main() {
  r := chi.NewRouter()
  r.Use(middleware.Logger)
  r.Get("/", func(w http.ResponseWriter, r *http.Request) {
    w.Write([]byte("welcome"))
  })
  http.ListenAndServe(":3000", r)
}
\end{minted}

\bigskip
3. Tasks: \begin{enumerate}
\item add endpoint \verb|/hello|, that returns \verb|world|;
\item add support for adding name: \verb|/hello?name=natalia|, should return: \verb|Hello Natalia!|;
\item add support for language as a {\small GET} parameter;
\item move all the logic to a separate package \verb|greetings|;
\item Use one of the middlewares, e.g., \verb|SetHeader| or \verb|BasicAuth|.
\end{enumerate}

%%%%%%%%%%%%%%%%%
\section{Linters}

Linters are a part of the language:

\begin{markdown}
- gofmt
- goimport - gofmt + sorting imports
- govet - now executed with tests
\end{markdown}

To apply the fixes, use \verb|-w| flag:

\begin{minted}{text}
$ gofmt -w .
$ goimports -w *.go && goimports -w */*.go
\end{minted}

There are many linters out there, check, for e.g., \href{https://github.com/golangci/awesome-go-linters}{awesome-go-linter page}.

You should call the linters as part of your {\small CI}/{\small CD} pipeline:

\begin{markdown}
- linter
- test
- integration-test
\end{markdown}

%%%%%%%%%%%%%%%%%%%%%
\section{References}

\begin{markdown}
- [golang.com/doc/effective_go.html](https://golang.com/doc/effective_go.html);
- [devs.cloudimmunity.com/gotchas-and-common-mistakes-in-go-golang](http://devs.cloudimmunity.com/gotchas-and-common-mistakes-in-go-golang).
\end{markdown}
\end{document}
